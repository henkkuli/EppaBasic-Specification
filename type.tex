\subsection{Type}
\label{type}
EppaBasic type system consist of language defined types, user defined types and derived types.
Language defined base types are \hyperref[type:Integer]{Integer},
\hyperref[type:Number]{Number}, \hyperref[type:String]{String} and \hyperref[type:Boolean]{Boolean}.
Users can define their own types using \hyperref[tld:structdef]{structures}.
Derived types are types, which are derived from other types,
like \hyperref[type:Array]{arrays} and \hyperref[type:Function]{function types}.

\subsubsection{Integer}
\label{type:Integer}
Integers are one of the base types in EppaBasic.
They are always copyable, unless otherwise specified at a function call.
Integers represent arbitrary precision integral numbers.

\subsubsection{Number}
\label{type:Number}
Numbers are one of the base types in EppaBasic.
They are always copyable, unless otherwise specified at a function call.
Numbers represent a 64-bit IEEE 754 double-precision floating point numbers.

\subsubsection{Double}
\label{type:Double}
Double is an alias for \hyperref[type:Number]{Number}.

\subsubsection{Boolean}
\label{type:Boolean}
Booleans are one of the base types in EppaBasic.
They are always copyable, unless otherwise specified at a function call.
Booleans represent either $Ture$ or $False$.
EppaBasic defines two global constants for these values: $True$ and $False$.

\subsubsection{String}
\label{type:String}
Strings are one of the base types in EppaBasic.
They are always copyable, unless otherwise specified at a function call.
Strings represent Unicode strings.

\subsubsection{Array}
\label{type:Array}
Strings are one of the base types in EppaBasic.
They are always copyable, unless otherwise specified at a function call.
Arrays represent an array (or table or list) of values of some specified type.

Arrays are created using the following syntax:
\begin{grammar}
<type>\sub{base}[<expression>\sub{1}, <expression>\sub{2}, ... <expression>\sub{n}]
\end{grammar}
This defines a n-dimensional array ($n\in[1,2,3,...]$).
Each expression defines the size of the corresponding dimension.
Each expression must be cast-able to \hyperref[type:Integer]{Integer}.
Indices in the i$^{th}$ dimension belong to $[0,1,2,...,size_i]$.
The type of items in arrays are $base$.

Note that the base type can be an array.

\subsubsection{Function}
\label{type:Function}
The compiler should implement function types,
though the exact syntax is not yet defined.
