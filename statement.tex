\subsection{Statement}
\label{statement}
A statement is the basic unit of source code.
Generally one statement occupies one line,
though in some cases multiple statements can be on the same line
and mutually one statement can be divided to multiple lines.
The division of statement to multiple lines can happen
at any time using \token{\string_}-token between individual tokens.

\subsubsection{Variable Definition}
\label{stat:vardes}
A variable definition defines a variable.
The defined variable has a \hyperref[scope]{scope}.
The definition must have either a \hyperref[type]{type} or an \hyperref[expr]{expression} or both.
If only expression is defined, the type of the variable is the type of the \hyperref[expr]{expression}.
If both \hyperref[type]{type} and \hyperref[expr]{expression} are defined, the \hyperref[expr]{expression} \hyperref[type]{type} is casted to \hyperref[type]{type}.

\begin{grammar}
Dim <identifier> As <type>
Dim <identifier> As <type> = <expression>
Dim <identifier> = <expression>
\end{grammar}

\subsubsection{Variable Assignment}
\label{stat:varass}
A variable assignment assigns the value of evaluated expression to the variable $name$
which is defined earlier in current scope or in one of its parent scopes.
The assignment follows \hyperref[copying]{copying rules}.

\begin{grammar}
<identifier>\sub{name} = <expression>
\end{grammar}

\subsubsection{For}
\label{stat:for}
A for statement loops its $body$ exactly $(end - start + 1) / step$ times
unless the loop is explicitly exited using \token{break}.
At the beginning of each iteration,
the $index$ is computed using $start + i \cdot step$,
where $i \in [0, 1, 2 ...]$ is the number of iteration.
If no $step$ is specified, $1$ is assumed.

$start$, $end$ and $step$ must all be cast-able to \hyperref[type:number]{number}.
The index cannot be set with assignment.

\begin{grammar}
For <identifier>\sub{index} = <expression>\sub{start} To <expression>\sub{end} [Step <expression>\sub{step}]
~~~~<block>\sub{body}
Next <identifier>\sub{index}
\end{grammar}

\subsubsection{If}
\label{stat:if}
A if statement is a conditional branch.
All $cond$-expressions must evaluate to \hyperref[type:boolean]{boolean}.
First the $cond\_1$ is evaluated.
If it is true,
$branch\_2$ the execution is transferred to $branc\_1$.
Otherwise either next $cond$ in is evaluated if such exists
or the execution is transferred to the $branch\_else$ if it exists.
If neither exists the execution is transferred to the statement following $End If$.
The same happens when any of the $branch$es ends.

All $cond$s must be cast-able to boolean.

\begin{grammar}
If <expression>\sub{cond\_1} Then
~~~~<block>\sub{branch\_1}
[
Else If <expression>\sub{cond\_2} Then
~~~~<block>\sub{branch\_2}
]
[
Else If <expression>\sub{cond\_3} Then
~~~~<block>\sub{branch\_3}
]
...
[
Else
~~~~<block>\sub{branch\_else}
]
End If
\end{grammar}

\subsubsection{Do-Loop}
\label{stat:do}
A Do-Loop statement executes its content repeatedly.
If condition is defined in the beginning,
the condition is first checked before entering the loop.
If the condition is in the end of the loop,
the $block$ is executed once before evaluating $cond$.
If no condition is defined,
the loop repeats itself forever.

If the condition type is $While$,
the $block$ is executed as long as the $cond$ evaluates $True$.
If the condition type is $Until$
the $block$ is executed as long as the $cond$ evaluates $False$.

The $cond$ must be cast-able to boolean.
Only one condition may exist in one Do-Loop.

\begin{grammar}
Do [While|Until <expression>\sub{cond}]
~~~~<block>
Loop [While|Until <expression>\sub{cond}]
\end{grammar}

\subsubsection{Return}
\label{stat:return}

\subsubsection{Comment}
\label{stat:comment}

\subsubsection{Include}
\label{stat:include}