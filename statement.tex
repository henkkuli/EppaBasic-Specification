\subsection{Statement}
\label{statement}
A statement is the basic unit of source code.
Generally one statement occupies one line,
though in some cases multiple statements can be on the same line
and mutually one statement can be divided to multiple lines.
The division of statement to multiple lines can happen
at any time using \token{\string_}-token between individual tokens.

\subsubsection{Variable Definition}
\label{stat:vardes}
A variable definition defines a variable.
The defined variable has a \hyperref[scope]{scope}.
The definition must have either a \hyperref[type]{type} or an \hyperref[expr]{expression} or both.
If only expression is defined, the type of the variable is the type of the \hyperref[expr]{expression}.
If both \hyperref[type]{type} and \hyperref[expr]{expression} are defined, the \hyperref[expr]{expression} \hyperref[type]{type} is casted to \hyperref[type]{type}.

\begin{grammar}%
Dim <identifier> As <type>
Dim <identifier> As <type> = <expression>
Dim <identifier> = <expression>
\end{grammar}

\subsubsection{Variable Assignment}
\label{stat:varass}
A variable assignment assigns the value of evaluated expression to the variable $name$
which is defined earlier in current scope or in one of its parent scopes.
The assignment follows \hyperref[copying]{copying rules}.

\begin{grammar}%
<identifier>\sub{name} = <expression>
\end{grammar}

\subsubsection{For}
\label{stat:for}
A for statement loops its $body$ exactly $\lfloor(end - start + 1) / step\rfloor$ times
unless the loop is explicitly exited using \token{break}.
At the beginning of each iteration,
the $index$ is computed using $start + i \cdot step$,
where $i \in [0, 1, 2 ...]$ is the number of iteration.
If no $step$ is specified, $1$ is assumed.

$start$, $end$ and $step$ must all be cast-able to \hyperref[type:Number]{number}.
The index cannot be set with assignment.

\begin{grammar}%
For <identifier>\sub{index} = <expression>\sub{start} To <expression>\sub{end} [Step <expression>\sub{step}]
~~~~<block>\sub{body}
Next <identifier>\sub{index}
\end{grammar}

\subsubsection{If}
\label{stat:if}
A if statement is a conditional branch.
All $cond$-expressions must evaluate to \hyperref[type:Boolean]{boolean}.
First the $cond\_1$ is evaluated.
If it is true, the execution is transferred to $branc\_1$.
Otherwise either next $cond$ in is evaluated if such exists
or the execution is transferred to the $branch\_else$ if it exists.
If the second $cond$ evaluates true, the excecution is transferred to $branch\_2$.
Otherwise next $cond$ is checked.
If neither exists the execution is transferred to the statement following $End If$.
The same happens when any of the $branch$es ends.

All $cond$s must be cast-able to boolean.

\begin{grammar}%
If <expression>\sub{cond\_1} Then
~~~~<block>\sub{branch\_1}
[
Else If <expression>\sub{cond\_2} Then
~~~~<block>\sub{branch\_2}
]
[
Else If <expression>\sub{cond\_3} Then
~~~~<block>\sub{branch\_3}
]
...
[
Else
~~~~<block>\sub{branch\_else}
]
End If
\end{grammar}

\subsubsection{Do-Loop}
\label{stat:do}
A Do-Loop statement executes its content repeatedly.
If condition is defined in the beginning,
the condition is first checked before entering the loop.
If the condition is in the end of the loop,
the $block$ is executed once before evaluating $cond$.
If no condition is defined,
the loop repeats itself forever.

If the condition type is $While$,
the $block$ is executed as long as the $cond$ evaluates $True$.
If the condition type is $Until$
the $block$ is executed as long as the $cond$ evaluates $False$.

The $cond$ must be cast-able to boolean.
Only one condition may exist in one Do-Loop.

\begin{grammar}%
Do [While|Until <expression>\sub{cond}]
~~~~<block>
Loop [While|Until <expression>\sub{cond}]
\end{grammar}

\subsubsection{Return}
\label{stat:return}

\subsubsection{Comment}
\label{stat:comment}

\subsubsection{Include}
\label{stat:include}


\subsection{Top Level Definition}
\label{tld}
A Top Level Definition is a special case of \hyperref[statement]{statement}:
it can only be located at the top level of a file.

\subsubsection{Function Definition and Sub-program Definition}
\label{tld:funcdef}
User defined functions and sub-programs don't differ by that much:
the only different is that the function can and must return a value.
All paths to the end of the block in a function must contain a \hyperref[stat:return]{return} statement
with appropriate type.

Both functions and sub-programs can take arbitrary number of parameters.
These parameters have types.
Parameters are copied by default.
If $ByRef$ is defined, the parameter may be passed as a reference,
though this must also be explicitly stated at the call site.

There may be multiple functions and sub-programs with the same name
as long as they have different parameter signatures, i.e.
they differ by number of parameters or by at least one parameter type.

\begin{grammar}%
Function <identifier>([[<ByRef>] <identifier>\sub{param} As <type>]*) As <type>
~~~~<block>
End Function

Sub <identifier>([<identifier>\sub{param} As <type>]*)
~~~~<block>
End Sub
\end{grammar}


\subsubsection{Structure Definition}
\label{tld:structdef}
