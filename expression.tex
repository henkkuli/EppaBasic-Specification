\subsection{Expression}
\label{expr}

\subsubsection{Evaluation Order}
\label{evalord}
The binary operators in the expressions are evaluated in the following order:

\begin{threeparttable}
\begin{tabular}{l | l | l}
    Operator     & Priority & Chain-able \\
    \hline
    $Or$         & 0        & Yes        \\
    $And$        & 1        & Yes        \\
    $Xor$        & 2        & Yes        \\
    $=$          & 3        & No         \\
    $<>$         & 3        & No         \\
    $<$          & 3        & No         \\
    $<=$         & 3        & No         \\
    $>$          & 3        & No         \\
    $>=$         & 3        & No         \\
    $\&$         & 4        & Yes        \\
    $+$          & 5        & Yes        \\
    $-$          & 5        & Yes        \\
    $/$          & 6        & Yes        \\
    $\backslash$ & 6        & Yes        \\
    $\ast$       & 6        & Yes        \\
    $Mod$        & 6        & Yes        \\
    $\wedge$     & 6        & Yes        \\
\end{tabular}
\end{threeparttable}

Unary operators take the highest precedence
so they are executed before any binary operator:

\begin{threeparttable}
\begin{tabular}{l | l}
    Operator       & Position \\
    \hline
    $-$            & Prefix   \\
    $Not$          & Prefix   \\
    $[expression]$ & Postfix  \\
\end{tabular}
\end{threeparttable}

Chain-ability means that multiple operators of that type can be chained, i.e.
$1 + 2 + 3$, but not $a = b = c$.
All chain-able operators are executed from left to the right, i.e.
$1+2+3+4$ is same as $((1+2)+3)+4$.

Before executing the operator, both of its sides are evaluated.
The evaluation order can be controlled with parenthesis, i.e.
$(1+2)*3$ evaluates to $9$.
Compare to $1+2*3$ which evaluates to $7$.

\subsubsection{Resolving Types}
\label{typeresolve}
\newcommand\type[1]{\hyperref[type:#1]{#1}}

Types of binary operators are resolved according to the following table:

\begin{threeparttable}
    \begin{tabular}{l | l}
        Binary Operator & Type \\
        \hline
        $And$, $Or$, $Xor$ &
        \begin{tabular}{l}
        $(\type{Boolean},\type{Boolean})\mapsto\type{Boolean}$\\
        $(\type{Integer},\type{Integer})\mapsto\type{Integer}$\\
        \end{tabular}\\
        \hline
        
        $=$, $<>$, $<$, $<=$, $>$, $>=$ &
        \begin{tabular}{l}
        $(\type{Number},\type{Number})\mapsto\type{Number}$\\
        $(\type{Integer},\type{Integer})\mapsto\type{Integer}$\\
        $(\type{String},\type{String})\mapsto\type{String}$\\
        \end{tabular}\\
        \hline
        
        $\&$ &
        \begin{tabular}{l}
        $(\type{String},\type{String})\mapsto\type{String}$\\
        \end{tabular}
        \footnotemark[1]\\
        \hline
        
        $+$, $-$, $*$, $\wedge$ &
        \begin{tabular}{l}
        $(\type{Number},\type{Number})\mapsto\type{Number}$\\
        $(\type{Integer},\type{Integer})\mapsto\type{Integer}$\\
        \end{tabular}
        \footnotemark[2]\\
        \hline
        
        $/$ &
        \begin{tabular}{l}
        $(\type{Number},\type{Number})\mapsto\type{Number}$\\
        \end{tabular}
        \footnotemark[3]\\
        \hline
        
        $\backslash$, $Mod$ &
        \begin{tabular}{l}
        $(\type{Integer},\type{Integer})\mapsto\type{Integer}$\\
        \end{tabular}
        \footnotemark[4]\\
    \end{tabular}
    \begin{tablenotes}
        \item[1] If other types are present, they are first casted to $\type{String}$.
        \item[2] If $\type{Number}$ and $\type{Integer}$ are mixed together in the expression,
                 both sides are first casted to $\type{Number}$.
        \item[3] Performs double-division.
                If other types are specified, they are first casted to $\type{Number}$.
        \item[4] Performs integer-division.
                 If other types are specified, they are first casted to $\type{Integer}$.
    \end{tablenotes}
\end{threeparttable}\\

Types of unary operators are resolved according to the following table:

\begin{threeparttable}
    \begin{tabular}{l | l}
        Unary Operator & Type \\
        \hline
        $-$ &
        \begin{tabular}{l}
        $\type{Number}\mapsto\type{Number}$\\
        $\type{Integer}\mapsto\type{Integer}$\\
        \end{tabular}\\
        \hline
        
        $Not$ &
        \begin{tabular}{l}
        $\type{Boolean}\mapsto\type{Boolean}$\\
        \end{tabular}\\
        \hline
        
        $[expression]$ &
        \begin{tabular}{l}
        $\type{Array}~of~T\mapsto T$\\
        \end{tabular}\\
    \end{tabular}
\end{threeparttable}

\let\type\undefined